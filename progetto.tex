% !TEX TS-program = pdflatex
% !TEX encoding = UTF-8 Unicode

% This is a simple template for a LaTeX document using the "article" class.
% See "book", "report", "letter" for other types of document.

\documentclass[11pt]{article} % use larger type; default would be 10pt

\usepackage[utf8]{inputenc} % set input encoding (not needed with XeLaTeX)

%%% Examples of Article customizations
% These packages are optional, depending whether you want the features they provide.
% See the LaTeX Companion or other references for full information.

%%% PAGE DIMENSIONS
\usepackage{geometry} % to change the page dimensions
\geometry{a4paper} % or letterpaper (US) or a5paper or....
% \geometry{margin=2in} % for example, change the margins to 2 inches all round
% \geometry{landscape} % set up the page for landscape
%   read geometry.pdf for detailed page layout information

\usepackage{graphicx} % support the \includegraphics command and options

% \usepackage[parfill]{parskip} % Activate to begin paragraphs with an empty line rather than an indent

%%% PACKAGES
\usepackage{booktabs} % for much better looking tables
\usepackage{array} % for better arrays (eg matrices) in maths
\usepackage{paralist} % very flexible & customisable lists (eg. enumerate/itemize, etc.)
\usepackage{verbatim} % adds environment for commenting out blocks of text & for better verbatim
\usepackage{subfig} % make it possible to include more than one captioned figure/table in a single float
% These packages are all incorporated in the memoir class to one degree or another...

%%% HEADERS & FOOTERS
\usepackage{fancyhdr} % This should be set AFTER setting up the page geometry
\pagestyle{fancy} % options: empty , plain , fancy
\renewcommand{\headrulewidth}{0pt} % customise the layout...
\lhead{}\chead{}\rhead{}
\lfoot{}\cfoot{\thepage}\rfoot{}

%%% SECTION TITLE APPEARANCE
\usepackage{sectsty}
\allsectionsfont{\sffamily\mdseries\upshape} % (See the fntguide.pdf for font help)
% (This matches ConTeXt defaults)

%%% ToC (table of contents) APPEARANCE
\usepackage[nottoc,notlof,notlot]{tocbibind} % Put the bibliography in the ToC
\usepackage[titles,subfigure]{tocloft} % Alter the style of the Table of Contents
\renewcommand{\cftsecfont}{\rmfamily\mdseries\upshape}
\renewcommand{\cftsecpagefont}{\rmfamily\mdseries\upshape} % No bold!

%%% END Article customizations

%%% The "real" document content comes below...

\usepackage{tabulary}

\newcommand{\AnalisiSituazioneAttualeTabella}[5]{
	\begin{tabulary}{\textwidth}{|C|C|}
		\hline
		Settore & #1 \\ \hline
		Attività & #2 \\ \hline
		Situazione informativa & #3 \\ \hline
		Modalità operativa & #4 \\ \hline
		Obiettivo & #5 \\
		\hline
	\end{tabulary}
}

\title{Fattibilità e Vision}
\author{The Author}
%\date{} % Activate to display a given date or no date (if empty),
         % otherwise the current date is printed 

\begin{document}
\maketitle

\section{Introduzione}

Il nostro obiettivo è quello di rendere più funzionale e pratica la gestione della piscina olimpica di Riccione.

Attualmente la gestione degli inscritti è ancora manuale, il che comporta lente e ripetitive operazionali manuali riguardanti la collezione delle quote associative, il rinnovo delle tessere e l'inscrizione di nuovi soci.

La piscina olimpica è la piscina usata dagli altleti della nazionale durante gli ordinari allenamenti, ma spesso sorgono conflitti tra le necessità di allenamento degli altleti e la fruizione della piscina da parte dei normali utenti.

La piscina, essendo una delle più grandi in italia, è spesso luogo designato per ospitare varie manifestazioni, l'operazione di prenotazione della struttura però è ancora manuale. 
La prenotazione deve essere fatta almeno due mesi prima dell'evento, in oltre non ci devono essere conflitti con le necessittà della FIN, ovviamente la piscina deve essere libera e prima di confermare la prenotazione si attende l'acconto della caparra.



\section{Obiettivi (B)}

Il nostro applicativo quindi si svilupperà in modo tale da risolvere le tre principali criticità analizzate nella Introduzione.

Parte del sistema informativo verrà finanziata da ACQUALIS, nota società di distribuzione di prodotti relativi al nuoto, che a fronte dell'investimento richiede un accesso privileggiato ai fruitori della piscina.

\begin{enumerate}
	\item Semplice ed automatica gestione degli inscritti con automatico rinnovo delle quote via pagamento online.  

	\item Possibilità di prenotare corsie per gli allenamenti, sia da parte della FIN che da parte dei normali fruitori della piscina.

	\item Possibilità di prenotare l'intera struttura per manifestazioni ed eventi.

	\item Creazione di store online per permettere ad ACQUALIS di raggiungere i propri clienti in modo efficacie.

\subsection{Stackeholder}

\begin{itemize}
	\item Ente gestore della piscina
	\begin{itemize}
		\item Gestore Eventi
		\item Gestore Finanze
		\item Personale amministrativo
	\end{itemize}
	\item FIN
	\begin{itemize}
		\item Allenatori
		\item Atleti
		\item Personale organizzativo
	\end{itemize}
	\item Privati fruitori della piscina
	\item ACQUALIS
\end{itemize}

\end{enumerate}

\section{Analisi della situazione attuale}

\AnalisiSituazioneAttualeTabella
	{Area Rapporti con il Pubblico}
	{Gestione degli inscritti alla piscina, controlli sul pagamento delle quote e rinnovo dei pagamenti}
	{Computer con suite di office automation}
	{Ad inizio settimana a tutti gli inscritti a cui scaderà la quota entro la fine della settimana viene mandata una email ricordando di pagare la quota di persona in piscina}
	{Automaticamente rinnovare la quota di iscrizione con pagamenti in via telematica}

\AnalisiSituazioneAttualeTabella
	{Area Organizzazione Interna}
	{Gestione dei turni per la fruizione delle corsie di allenamento}
	{Nessun sistema informatico.}
	{I bagnini decidono autonomamente come allocare le corsie bilanciando i vari bisogni del momento.}
	{Possibilità di prenotazione delle corsie in anticipo così da permettere alla FIN di allenarsi in modo consistente e continuo, evitare il sovraffolamento della piscina.}

\AnalisiSituazioneAttualeTabella
	{Area Eventi Straordinari}
	{Gestione, organizazione e supporto di grandi eventi}
	{Suite di office automation}
	{Tutti i contatti sono gestiti dal personale amministrativo che dopo una richiesta di prenotazione via email controlla che la data sia disponibile, in caso affermativo richiede il pagamento di una caparra e, a pagamento avvenuto, alloca la struttura per la data richiesta.}
	{Automatizzare il processo di prenotazione della struttura.}

\AnalisiSituazioneAttualeTabella
	{Area Rapporti con Esterni}
	{Gestione dei rapporti con ACQUALIS}
	{Nessuna, lo sviluppo del sistema informativo rappresenta l'inizio della collaborazione tra la l'Ente della piscina e ACQUALIS}
	{Nessuna}
	{Creare un store web-based dove i fruitori della piscina possono accedere ai prodotti offerti da ACQUALIS}

\section{Descrizione dei problemi}

I problemi riscontrati al momento sono:

\begin{itemize}

	\item Operazione amministrative svolte in modo lento, ripetitivo e prono ad errori e non integrata.
	\item Conflitti tra i privati fruitori della piscina e la FIN per l'uso delle corsie di allenamento.

\end{itemize}

\section{Soluzione proposta}

Proponiamo lo sviluppo di applicativi web-based per:

\begin{itemize}
	\item Gestione automatica rinnovo quote inscrizione
	\item Allocazione automatica delle corsie di allenamento
	\item Automatizzazione per la prenotazione della intera struttura
	\item Shop online a servizio dei membri della piscina gestito da ACQUALIS
\end{itemize}	

\section{Driver (B)}

\section{Struttura dell' organizzazione (O)}

\section{Requisiti funzionali}
% con tabelle

\section{Requisiti Non funzionali}

\begin{description}
	\item[Prestazioni] Supporto per 800 connesioni simultanee
	\item[Qualità dei dati] Dati aggiornati al completamento delle richieste
	\item[Efficienza] Massimo tempo di risposta ad ogni richiesta web 2sec, 90 percentile massimo a 0.5sec
	\item[Efficacia] Netto snellimento delle pratiche burocratiche, riduzione dei conflitti nel piano vasca, riduzione del sovraffolamento.
	\item[Robustezza] Possibile downtime programmato ogni giorno tra le 01:00 e le 02:00, affidabilità con 0.999 \%
	\item[Sicurezza] Non memorizzazione di dati sensibile degli utenti
	\item[Disponibilità] Servizio consultabile 24/7/365
	\item[Usabilità] Usabilità da dispositivi mobile e fissi, massimo 6 click per raggiungere ogni funzione, massimo 2 click per raggiungere le 5 funzioni più usate.
\end{description}

%sicurezza, performace, usabilità

\section{Gantt}

\section{Architettura}
%disegno

\section{Architettura market-part-system (A)}

\end{document}

% !TEX TS-program = pdflatex
% !TEX encoding = UTF-8 Unicode

% This is a simple template for a LaTeX document using the "article" class.
% See "book", "report", "letter" for other types of document.

\documentclass[11pt]{article} % use larger type; default would be 10pt

\usepackage[utf8]{inputenc} % set input encoding (not needed with XeLaTeX)

%%% Examples of Article customizations
% These packages are optional, depending whether you want the features they provide.
% See the LaTeX Companion or other references for full information.

%%% PAGE DIMENSIONS
\usepackage{geometry} % to change the page dimensions
\geometry{a4paper} % or letterpaper (US) or a5paper or....
% \geometry{margin=2in} % for example, change the margins to 2 inches all round
% \geometry{landscape} % set up the page for landscape
%   read geometry.pdf for detailed page layout information

\usepackage{graphicx} % support the \includegraphics command and options

% \usepackage[parfill]{parskip} % Activate to begin paragraphs with an empty line rather than an indent

%%% PACKAGES
\usepackage{booktabs} % for much better looking tables
\usepackage{array} % for better arrays (eg matrices) in maths
\usepackage{paralist} % very flexible & customisable lists (eg. enumerate/itemize, etc.)
\usepackage{verbatim} % adds environment for commenting out blocks of text & for better verbatim
\usepackage{subfig} % make it possible to include more than one captioned figure/table in a single float
% These packages are all incorporated in the memoir class to one degree or another...

%%% HEADERS & FOOTERS
\usepackage{fancyhdr} % This should be set AFTER setting up the page geometry
\pagestyle{fancy} % options: empty , plain , fancy
\renewcommand{\headrulewidth}{0pt} % customise the layout...
\lhead{}\chead{}\rhead{}
\lfoot{}\cfoot{\thepage}\rfoot{}

%%% SECTION TITLE APPEARANCE
\usepackage{sectsty}
\allsectionsfont{\sffamily\mdseries\upshape} % (See the fntguide.pdf for font help)
% (This matches ConTeXt defaults)

%%% ToC (table of contents) APPEARANCE
\usepackage[nottoc,notlof,notlot]{tocbibind} % Put the bibliography in the ToC
\usepackage[titles,subfigure]{tocloft} % Alter the style of the Table of Contents
\renewcommand{\cftsecfont}{\rmfamily\mdseries\upshape}
\renewcommand{\cftsecpagefont}{\rmfamily\mdseries\upshape} % No bold!

%%% END Article customizations

%%% The "real" document content comes below...

\usepackage{tabulary}
\usepackage{pgfgantt}

\newcommand{\AnalisiSituazioneAttualeTabella}[5]{
	\begin{tabulary}{\textwidth}{|C|C|}
		\hline
		Settore & #1 \\ \hline
		Attività & #2 \\ \hline
		Situazione informativa & #3 \\ \hline
		Modalità operativa & #4 \\ \hline
		Obiettivo & #5 \\
		\hline
	\end{tabulary}
}

\title{Fattibilità e Vision}
\author{The Author}
%\date{} % Activate to display a given date or no date (if empty),
         % otherwise the current date is printed 

\begin{document}
\maketitle

\section{Obiettivi (B)}

L'obbiettivo del nostro progetto è quello di migliorare il sistema informativo già esistente della piscina di Riccione apportando buove funzionalità al sistema.

L'intervento si concentrerà su:

\begin{enumerate}
	\item Gestione degli iscritti, automatizzando le procedure di rinnovo della quota associativa
	\item Gestione delle corsie della piscina, permettendo la prenotazione delle corsie in modo da evitare sovraffollamento
	\item Gestione degli eventi, automatizzando la procedura di affitto della intera struttura per eventi
\end{enumerate}

L'applicattivo avrà funzionalità di sostegno ai reparti: Amministrazione Attività Sportive,  Gestione Finanze e Rapporti con Enti Esterni.



\section{Introduzione}


La piscina è utilizzata dagli atleti olimpici della Federazione Italiana Nuoto (FIN) durante gli ordinari allenamenti ed essendo una delle migliori in Italia è spesso luogo designato per ospitare varie manifestazioni.

Parte del costo del sistema informativo verrà abbattuto tramite una partership con ACQUALIS, nota società di distribuzione di prodotti relativi al nuoto.

La partership prevederà l'introduzione di uno store all'interno del portale web della piscina dedicato esclusivamente ai prodotti ACQUALIS.

Si vogliono introdurre diverse funzionalità:
\begin{enumerate}
	\item Semplice ed automatica gestione degli iscritti con automatico rinnovo delle quote via pagamento online. 
	\item Possibilità di prenotare corsie per gli allenamenti, sia da parte della FIN che da parte dei normali fruitori della piscina.
	\item Possibilità di prenotare l'intera struttura per manifestazioni ed eventi.
	\item Creazione di store online integrato di ACQUALIS.
\subsection{Stackeholder}

\begin{itemize}
	\item Ente gestore della piscina
	\begin{itemize}
		\item Amministrazione Attività Sportive
		\item Gestione Finanze
		\item Rapporti con Enti Esterni
	\end{itemize}
	\item FIN
	\begin{itemize}
		\item Allenatori
		\item Atleti
		\item Personale organizzativo
	\end{itemize}
	\item Privati fruitori della piscina
	\item ACQUALIS
\end{itemize}


\section{Background}

Il sistema informativo della piscina è limitato a pochi personal computer con semplice gestione di basi di dati da suite di automazione.

Il sistema è gia integrato con un centro medico della zona che gestisce le visite mediche degli atleti.

\end{enumerate}

\section{Analisi della situazione attuale}

\AnalisiSituazioneAttualeTabella
	{Amministrazione Attività Sportive}
	{Gestione degli iscritti alla società, controlli sul pagamento delle quote e rinnovo dei pagamenti}
	{Computer con suite di office automation}
	{Ad inizio settimana a tutti gli iscritti a cui scaderà la quota entro la fine della settimana viene mandata una email ricordando di pagare la quota di persona in piscina}
	{Rinnovo della quota di iscrizione con possibilità di pagamenti in via telematica}

\AnalisiSituazioneAttualeTabella
	{Amministrazione Attività Sportive e Rapporti con Enti Esterni}
	{Gestione dei turni per la fruizione delle corsie di allenamento}
	{Nessun sistema informatico.}
	{I bagnini decidono autonomamente come allocare le corsie bilanciando i vari bisogni del momento.}
	{Possibilità di prenotazione delle corsie in anticipo via web così da evitare il sovraffolamento della piscina e permettere alla FIN di allenarsi in modo consistente e continuo.}

\AnalisiSituazioneAttualeTabella
	{Rapporti con Enti Esterni}
	{Gestione, organizazione e supporto di grandi eventi}
	{Suite di office automation}
	{Tutti i contatti sono gestiti dal personale amministrativo che dopo una richiesta di prenotazione via email controlla che la data sia disponibile, in caso affermativo richiede il pagamento di una caparra e, a pagamento avvenuto, alloca la struttura per la data richiesta. Viene quindi affisso un cartello in bacheca indicando la data in cui tutta la struttura è prenotata e inagibili per i normali fruitori.}
	{Automatizzare il processo di prenotazione della struttura e invio automatico di comunicazione telematica ai soci della società per comunicare l'inagibilità nel giorno programmato.}

\AnalisiSituazioneAttualeTabella
	{Rapporti con Enti Esterni}
	{Gestione dei rapporti con ACQUALIS}
	{Nessuna, lo sviluppo del sistema informativo rappresenta l'inizio della collaborazione tra la società e ACQUALIS}
	{Nessuna}
	{Creare un store web-based dove i fruitori della piscina possono accedere ai prodotti offerti da ACQUALIS}

\section{Descrizione dei problemi}

I problemi riscontrati al momento sono:

\begin{itemize}
	\item Sovraffolamento della piscina con utenze concentrate in pochi particolari orari
	\item Conflitti tra i privati fruitori della piscina e la FIN per l'uso delle corsie di allenamento.
	\item Operazione amministrative svolte in modo lento, ripetitivo, prono ad errori e non integrato.
	\begin{itemize}
		\item Problemi di integrazione tra le manifestazioni e il normale utilizzo della piscina
		\item Lentezza nel riscuotere le quote associative
	\end{itemize}
\end{itemize}

\section{Soluzione proposta}

Proponiamo lo sviluppo di applicativi web-based per:

\begin{itemize}
	\item Gestione rinnovo quote
	\item Possibilità di prenotazione di posti nelle corsie di allenamento
	\item Automatizzazione per la prenotazione della intera struttura
	\item Store ACQUALIS dedicato ai membri della piscina
\end{itemize}

\section{Driver (B)}

La adozione del sistema informativo è guidata da driver sia di efficacia che di efficienza.

\subsection{Efficacia}

La possibilità di prenotare posti e corsi di allenamento permette di evitare conflitti e malcontenti nel bordo vasca, oltre che permettere un più constante e continuo allenamento degli atleti della FIN

\subsection{Efficienza}

Molti features del Sistema Informativo migliorano l' efficienza della associazione.

\begin{description}
	\item[Automatica gestione del rinnovo delle quote:] la gestione informatizzata renderà molto più snello e veloce il lavoro del personale amministrativo, inoltre sarà possibile tenere meno soldi in cassa aumentando la sicurezza della struttura. 
	\item[Automatica prenotazione della intera struttura:] la gestione informatizzata eviterà che il pesonale continui a rispondere a richieste di persone poco interessate e permetterà di concentrare le risorse solo per gli enti che hanno già pagato la caparra e riuscendo a garantire un servizio migliore.
\end{description}

\section{Struttura dell' organizzazione (O)}

\section{Requisiti funzionali}

\begin{enumerate}
	\item La società richiede periodicamente ai soci il rinnovo delle quote associative via email, i soci pagano online oppure alla segreteria.
	\item La società dispone di un portale web dove i fruitori della piscina devono prenotarsi per sfruttare le corsie.
	\item La società dispone di un portale web nel quale è possibile noleggiare l'intera struttura per un periodo di tempo limitato, vengono forniti gli estremi per il pagamento della caparra e finalizzata la prenotazione.
	\item La società dispone di uno store online dedicato ad uso dei soci gestito da ACQUALIS
\end{enumerate}

\subsection{Requisiti funzionali Specifichi}

\begin{enumerate}
	\item La società richiede periodicamente ai soci il rinnovo delle quote associative via email, i soci pagano online oppure alla segreteria.
	\begin{enumerate}
		\item Ogni 24 ore uno script automatico viene lanciato in background
		\item Le informazioni riguardanti i soci la cui quota è in scadenza vengono automaticamente ritrovate
		\item Viene mandata una email richiedendo il rinnovo della quota con pagamento o telematico o manuale direttamente in sede
		\item Il servizio web della segreteria viene aggiornato in modo tale da evidenziare i soci che devono ancora pagare la quota
		\item Al pagamento della quota via online, il servizio web è automaticamente aggiornato per mostrare che il socio ha effettivamente rinnovato la quota
		\item Al pagamento della quota in sede, il personale può marcare che il socio ha rinnovato la propria quota.
	\end{enumerate}
	\item La società dispone di un portale web dove i fruitori della piscina devono prenotarsi per sfruttare le corsie.
	\begin{enumerate}
		\item I soci della piscina possono accedere all'area web riservato del portale web in cui è possibile prenotare un posto in una corsia per allenamento
		\item Il personale FIN può accedere all'interno della proprio area riservate del portale web in cui è possibile prenotare più corsie per gli allenamenti degli atleti
		\item La compilazione della richiesta è guidata dal sistema e permessa dopo autenticazione.
		\item I dati della prenotazione, previa verifica di disponibilità, sono automaticamente inseriti nel sistema
		\item Viene inviata una email per ricordare l'avvenuta prenotazione
		\item Personale autorizzato della società può consultare l'elenco delle prenotazioni per risolvere possibili equivoci a bordo vasca
	\end{enumerate}
	\item La società dispone di un portale web nel quale è possibile noleggiare l'intera struttura per un periodo di tempo limitato, vengono forniti gli estremi per il pagamento della caparra e finalizzata la prenotazione.
	\begin{enumerate}
		\item Il personale esterno qualificato che necessità di noleggiare la struttura può accedere ad un area del piattaforma web previa autenticazione
		\item Il prezzo e la disponibilità della struttura sono mostrati in un interfaccia web
		\item L'utente compila un questionario guidata dalla piattaforma
		\item Il sistema comunica via email il costo totale per noleggio, la caparra dovuta e gli estremi del pagamento
		\item Appena la caparra viene pagata il sistema registra la prenotazione
	\end{enumerate}
	\item La società dispone di uno store online dedicato ad uso dei soci gestito da ACQUALIS
	\begin{enumerate}
		\item I soci della società, previa autenticazione, hanno accesso ad uno store online fornito da ACQUALIS
		\item Lo store si interfaccia via programmatica alla piattaforma di ACQUALIS la quale comunica i prodotti da mostrare, i prezzi e le quantità disponibili
		\item All'acquisto di un prodotto il pagamento viene gestito completamente da ACQUALIS
		\item Dopo l'acquisito viene richiesto se si preferisce ricevere il prodotto a casa o direttamente in piscina, tale richiesta viene inoltrata ad ACQUALIS
		\item Nel caso in cui venga deciso di ricevere il prodotto in piscina, il personale viene avvertito e tiene in consegna il prodotto finchè il socio non lo richiede.
	\end{enumerate}
\end{enumerate}

\section{Requisiti Non funzionali}

\begin{description}
	\item[Prestazioni] Supporto per 800 connesioni simultanee
	\item[Qualità dei dati] Dati aggiornati al completamento delle richieste
	\item[Efficienza] Massimo tempo di risposta ad ogni richiesta web 2sec, 90 percentile massimo a 0.5sec
	\item[Efficacia] Netto snellimento delle pratiche burocratiche, riduzione dei conflitti nel piano vasca, riduzione del sovraffolamento.
	\item[Robustezza] Possibile downtime programmato ogni giorno tra le 01:00 e le 02:00, affidabilità con 0.999 \%
	\item[Sicurezza] Non memorizzazione di dati sensibile degli utenti
	\item[Disponibilità] Servizio consultabile 24/7/365
	\item[Usabilità] Usabilità da dispositivi mobile e fissi, massimo 6 click per raggiungere ogni funzione, massimo 2 click per raggiungere le 5 funzioni più usate.
\end{description}

%sicurezza, performace, usabilità

\section{Gantt}

% studio di fattibilità 11gg
% pianificazione 13gg
% identificazione e analisi requisiti 15gg
% definizione architettura 12gg
% implementazione 30gg
%definizione UI|UX 11gg
% implementazione UI|UX 30gg
% testing 12gg
% sviluppo ambiente 6gg
% gestione possibili cambiamenti 20gg
% gestione progetto 118gg

%\begin{ganttchart}{1}{12}
%	\gantttitle{2016}{52} \\
%	\gantttitlelist{1,...,52}{1} \\
%	\ganttgroup{Group 1}{1}{7} \\
%\end{ganttchart}

\section{Architettura}
%disegno

\section{Architettura market-part-system (A)}

\end{document}
